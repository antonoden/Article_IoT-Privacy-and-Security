\documentclass[article,a4paper]{IEEEtran}
\usepackage{lipsum}
\usepackage[backend=biber]{biblatex}
\usepackage{graphicx}

\addbibresource{refs.bib}
\title{Security and Privacy}
\author{
\IEEEauthorblockN{Anton Odén}\\
\IEEEauthorblockA{Dept. of Maths and Computer Science\\Karlstad University\\
651 88 KARLSTAD, Sweden}\\
anton.oden@outlook.com
}

\begin{document}

\maketitle

\begin{abstract}
    
\end{abstract}

\section{Introduction: The Growing Need for IIoT Security}
The groving interconnectivity between devices, sensors and cloud systems exposes industrial networks to various cyber security threats. Industry 4.0 goal with Internet of things connect all things create such great mass of data and datapoints being impossible by human to handle without advanced security algorithms. This article is based upon two recent articles providing insights into how technologies within big data analytics, deep learnig and edge computing can strengthen IIoT security. Discussing condidentiality, integrity, validity, authentication, access control etc.     
\newline\newline
The first paper, a "A Survey on Industrial Internet of Things Security: Requirements, Attacks, AI-Based Solutions, and Edge Computing Opportunities" by Bandar Alotaibi discuss key security challenges in IIoT ecosystems, highligting vulnerabilities across perception (end nodes), network and application layer. It is rich in examples being a survey of many papers and promotes implementing intrusion detection systems in edge computing locations for easier findings and blocking of maliouses software.   
\newline\newline
Second paper "Internet of Things Security Based on Big Data and Deep Learning" by Jian-Liang Wang and Ping Chen, focuses on how deep learning techniques can enhance IoT security. 

Both papers illustrate how AI-powered solutions can create more resilient industrial networks. This article examine their findings and explore how industries can integrate big-data deiven security to safeguard IIoT systems. 
\section{Security Challenges in Industrial IoT: An Overview}
As industries continue to implement IoT technologies the digital transformation also exposes vulnerabilities, making security a priotity within Industrial IoT (IIoT). Unlike traditional IT infrasturcture, IIoT enviroments consist of distributed devices of which many operates in remote industrial locations. This decentralization creates a wide attack surface being more difficult to monitor and defend against cyber threats. 
\newline\newline
Many industrial machinery also rely on outdated hardware and software, originally designed with less or no modern cybersecurity considerations. These systems often lack encryption, secure authentication and patching protocols, making them easy targets. Additionally, the absence of universal security standards complicates efforts to implement consistent protection across different IIoT deployments. Security breaches aren't all coming directly from external sources either. Insider threats, wheter intentional or accidental pose a big risk to IIoT systems. Weak access controls, improper credential managegment, and lack of employee cybersecurity awareness can lead to unauthorized access, data leaks, system failures, ransomware etc. 
\newline\newline
IIoT networks are also reliant or often integrate components from multiple vendors, introducing risks related to third-party software vulnerabilities and potential backdoor access. If a supplier experiences a security breach, attacker could use compromised informeration about that vendors devices to infiltrate industrial infrastructure.
\section{Types of Cyberattacks Threatening IIoT Systems}
Attackers exploit vulnerabilities across different layers of IIoT architecture, potentially causing operational disruption, financial losses, safety hazards and more. This section explores the most prominent cyberattacks targeting IIoT systems. 
\subsection{Perception Layer Attacks}
The perception layer consists of various objects, such as sensors, cameras, robots and smart meters. his layer is responsible of gathering of data/information for example vibrations, heat, acceleration, or humidity. The collected data is then transmitted to the network layer to be transported to an information processing system at the edge. Here follows some attacks targeting the perception layer.
\newline\newline
In a Node Capture Attack, the attacker physically obtain, replace or modify hardware. This act could lead to exposing sensitive information related to encryption or access keys and once the attacker got hold of this information the attacker could get control over the device and start targeting or cause harm to other devices in the network. 
\newline\newline
Sleep Deprivation attack is a type of DoS attack where the device isn't allowed to change stance to sleeping mode. A function end devices running on battery is dependent on to 

\subsection{Network Layer Attacks}
Denial of Service (DoS) attacks aim to overwhelm IIoT networks or devices with excessive traffic, making them slow or completely unresponsive. A sucessful DoS or Distributed Denial of Service (DDoS) attack can halt production lines, disable smart sensors or disrupt communication between critical infrastructure components.
\newline\newline
\subsection{Application Layer Attacks}

\section{Essential Security Requirements for IIoT Protection}

\section{The Role of AI in Industrial IoT Cybersecurity}

\section{Leveraging Edge Computing for Enhanced Security}

\section{Case Studies: Real-World Security Breaches in IIoT}

\section{Future Perspectives: Strengthening IIoT Defenses}

\section{Conclusion: Towards a Secure IIoT Ecosystem}

\printbibliography

\end{document}